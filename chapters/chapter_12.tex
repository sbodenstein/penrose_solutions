% --------------------------------------
% Chapter 12 Solutions
% --------------------------------------


\subsection{12.1 $\bigstar$}
One definition of the dimension of a space, is the \emph{minimum} number of seperate numbers required to uniquely fix the position of a point in that space. 
\subsection*{Method 1}
Take a book. One requires three numbers to specify where in 3-d space the centre of mass is. Now how many rotational degrees of freedom are there? Twist it around to see! There are three axes of rotation, one through the front cover, one through the spine and one through the top (if you are standing the book up).\\ \\ Now flip the book by 90 degrees about the front cover. Note the position of the book. Put it back to where it was, and try to get the book to the front-cover flip position by only turning it about the spine and top. You will see this can't be done! This shows that two rotation axes are not enough to define where the book is. And so specifying how much the book is rotated through each of these axes is another 3 numbers. And so it requires a minimum of 6 numbers to completely specify the position of the book, and so the dimension of the space of all the possible positions is 6. 

\subsection*{Method 2} 
A more elegant method. We first assume something obvious: if you tell me where 3 points are on a rigid body, I know exactly where the rest of the points are.\footnote{If this gives problems, think of the points as fixed hinges connected to the object. If there is one hinge, the rigid body can swivel in any direction about it. With another hinge, you can still rotate the object about an axis through both hinges. Add another one, and you can't move the body. It is stuck. } Two are obviously not enough. Now naively, we think it requires 9 coordinates to specify these 3 points in 3-d space. But we can do better. \\ \\ Let us our three points be $p_{cm}, p_2, p_3$. Fixing $p_{cm}$ requires 3 coordinates. Once this is fixed, $p_2$ can only be a fixed distance from $p_{cm}$, and so be on the surface of a sphere centred about $p_{cm}$. The sphere is 2-d. Once $p_2$ is fixed, $p_3$ can only be somewhere a fixed distance from both $p_{cm}$ and $p_2$. The set of points satisfying this condition is a circle, which is 1-d. And so the total dimension is 3+2+1=6.
