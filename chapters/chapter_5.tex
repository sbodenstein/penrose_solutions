% --------------------------------------
% Chapter 5 Solutions
% --------------------------------------

\subsection{5.5 $\bigstar \bigstar \bigstar$}
Let us first look at $e^a e^b$.
\begin{align*}
e^a e^b&=e^a(1+b+\frac{b^2}{2!}+\ldots)=e^a+e^a b+\frac{e^a b^2}{2!}+\ldots+\frac{e^a b^k}{k!}+\ldots\\
&=\sum^{\infty}_{n=0}\frac{a^n}{n!}+\sum^{\infty}_{n=0}\frac{a^nb}{n!} +\ldots+\sum^{\infty}_{n=0}\frac{a^n b^k}{n! k!}+\ldots\\
&=\sum_{k=0}^{\infty}\left(\sum^{\infty}_{n=0}\frac{a^n b^k}{n! k!}\right)
\end{align*}
Now consider $e^{a+b}$.We are given that the coefficient of $a^p b^q$ in $(a+b)^n$ is $\frac{n!}{p!q!}$. But we know that $p+q=n$. So let $p=n-q$, then  
\begin{equation}
(a+b)^n=\sum^{n}_{k=0}\frac{n!}{k!(n-k)!}a^{n-k}b^k
\end{equation}
Then we have that
\begin{align*}
e^{a+b}=\sum^{\infty}_{n=0}\frac{(a+b)^n}{n!}&=\sum^{\infty}_{n=0}\left(\sum^{n}_{k=0}\frac{1}{k!(n-k)!}a^{n-k}b^k\right)
\end{align*}
To see that this is the same as for $e^ae^b$, let us consider what the above sum means. Suppose we fix $k=i$. This will only exist when $i\leq n$. Then the sum becomes 
$$\sum^{\infty}_{n=i}\frac{1}{i!(n-i)!}a^{n-i}b^i=\sum^{\infty}_{q=0}\frac{1}{i!q!}a^q b^i$$
using $q\equiv n-i$. But the above must be summed over all $k$, not just for $k=i$. And as $n$ has range to $\infty$, so must $k$. So 
\begin{align*}
e^{a+b}=\sum^{\infty}_{n=0}\left(\sum^{n}_{k=0}\frac{1}{k!(n-k)!}a^{n-k}b^k\right)=\sum^{\infty}_{k=0}\sum_{q=0}^{\infty} \frac{1}{k!q!}a^q b^k=e^ae^b
\end{align*}


